\documentclass[a4paper]{article}
\usepackage[utf8]{inputenc}
\usepackage[T1]{fontenc}
\usepackage{lmodern}
\usepackage{graphicx}
\usepackage{bm}
\usepackage{hyperref}
\usepackage{amsmath}
\usepackage{amssymb}
\usepackage{MnSymbol}
\usepackage{units}
\usepackage{listings}


%macros
%\newcommand{\mod}{\text{ mod }}



\begin{document}
\title{Cryptography}
\section{Primes and Divisibility} 
\subsection{Division with Remainder}
$a$ and $b$ are integers and $b>0$ then there are integers $b,r$ for which $a = qb + r$ with $0 \leq r < b$

\subsection{Greatest Common Divisor}
$a,b \geq 0$ then $gcd(a,b) = c$ with $c$ being the largest integer that $c|a \wedge c|b$
\begin{equation}
	gcd(b,0) = b 
\end{equation}
if $p$ is prime then $gcd(a,p) = 1 or p$

if $a,b > 0$ Then integers $X,Y$ exist, such that $Xa+Yb=gcd(a,b)$


\section{Modular Arithmetic}
\subsection{Definition}
\begin{equation}
    \begin{split}
	[a \mod N] = [b \mod N] \\
	\text{is defined as: } a \equiv b \mod N
    \end{split}
    \label{modulo_definition}
\end{equation}

\subsection{properties}
Congruence modulo $N$ is an equivalent relation i.e.
\begin{equation}
    \begin{split}
	a \equiv a \mod N  \quad \forall  a \text{ (reflexive)} \\
	a \equiv b \mod N \quad\Rightarrow\quad b \equiv a \mod N \text{ (symmetric)}\\
	a \equiv b \mod N  \quad\wedge\quad  b \equiv c \mod N \quad\Rightarrow\quad a \equiv c \mod N \text{ (transitive)}
    \end{split}
    \label{modulo_equivalent_relation}
\end{equation}
Congruence modulo $N$ respects standard rules of arithmetic i.e.
\begin{equation}
    \begin{split}
	\text{if }a \equiv a' \mod N \quad\wedge\quad b \equiv b' \mod N \\
	\Rightarrow\quad ab \equiv a'b' \mod N \quad\wedge\quad a+b \equiv a'+b' \mod N
    \end{split}
    \label{modulo_arithmetic_rules}
\end{equation}

\textbf{but} does not respect division
\begin{equation}
    ab \equiv cb \mod N \nRightarrow a \equiv c \mod N 
    \label{modulo_arithmetic_division}
\end{equation}

except if $b$ is invertible i.e. it exists $b^{-1}$ that $bb^{-1} \equiv  1 \mod N$ 
\begin{equation}
    \begin{split}
	ab \equiv cb \mod N \\
	\Rightarrow abb^{-1} \equiv cbb^{-1} \mod N \\
	\Rightarrow a \equiv b \mod N
    \end{split}
    \label{modulo_arithmetic_multiplication_invertible}
\end{equation}

\subsubsection{Invertible Integers}
Let $a,N$ be integers, with $N > 1$. Then $a$ is invertible $\mod N$ if and only if $gcd(a,N) =1$

\subsubsection{Example}
It is possible to first reduce and then to add or multiply, instead of add or multiply and then reduce.
e.g 
\begin{equation*}
    \begin{split}
	[134092348\cdot134938402 \mod 100] \stackrel{slow}{=} [18094207159547896\mod 100] = 96 \\
	\Leftrightarrow[134092348\cdot134938402 \mod 100] = [134092348 \mod 100]\cdot[134938402 \mod 100] = 48\cdot2 = 96 
    \end{split}
\end{equation*}
But it is not possible to reduce modulo expressions
\begin{equation*}
    \begin{split}
	[15 \cdot 2 = 30] \equiv[3 \cdot 2 = 6] \mod 24 \\
	15 \nequiv 3 \mod 24
    \end{split}
\end{equation*}

\section{Groups}
\subsection{Definition}
A group is a set $\mathbb{G}$ with a binary operation $\circ$ with the following conditions
\begin{equation}
    \begin{split}
	\forall g,h \in \mathbb{G}, g \circ h \in \mathbb{G} \text{ (closure)}\\
	\exists e \in \mathbb{G} \text{ such that } \forall g \in \mathbb{G}, e \circ g = g \circ e = g \text{ (identity)} \\
	\forall g \in \mathbb{G} \exists h \in \mathbb{G} \text{ such that } g \circ h = e = h \circ g \text{ (inverse)}\\
	\forall g_1,g_2,g_3 \exists \mathbb{G} (g_1 \circ g_2) \circ g_3 = g_1 \circ (g_2 \circ g_3) \text{ (associativity)}
    \end{split}
    \label{group_definition}
\end{equation}
If for a Group $\mathbb{G}$ with the operation $\circ$ also the following holds true 
\begin{equation}
    \forall g,h \in \mathbb{G}, g \circ h = h \circ g \text{ (commutativity)}
    \label{group_abelian_defintion}
\end{equation}
\section{Number theory}
\subsection{Definition}
The set \[\mathbb{Z}_n^* := \{a \in \mathbb{Z}_n | \text{gcd}(a,n) = 1\}\] is called multiplicative group of $\mathbb{Z}_n$ and the function $\varphi(n) = |\mathbb{Z}_n|$ is called Euler-$\varphi$-function.
\begin{itemize}
	\item $\varphi(p) = p-1$ for any prime p
	\item $\mathbb{Z}_n^*$ is a multiplicative abelian group; it holds that $\text{gcd}(a,n) = 1 \leftrightarrow $ There exists an inverse s of a, i.e. $a \cdot s \equiv 1 \text{mod} n$
	\item notation: $(a,n) := \text{gcd}(a,n)$
\end{itemize}

\textbf{Euler's theorem:}

Let $a \in \mathbb{Z}_n^*$
\[ a^{\varphi(n)} \equiv 1 \text{mod} n  \]


\textbf{Fermat's little theorem:}

p is prime and $(a,p) = 1$, then $a^{p-1} \equiv 1 \text{mod}p$

\subsection{Fermat Primality Test (FPT)}

Select $a \in {2,\dots,n-1}$ at random and decide:

\[ \begin{cases} n \text{ is prime, if } a^{n-1} \equiv 1 \text{ mod } n \\ n \text{ is composite, otherwise} \end{cases} \]

The FPT is not suitable as a factorization test, because it just shows if n owns the property of being composite or not. 

Example for $n = 341 = 11 \cdot 31$ :

\[ \textbf{2}^340 \equiv 1 \text{ mod } 341 \]

even if n is composite. But:
\[ \textbf{3}^340 \equiv 56 \text{ mod } 341 \]

\subsection{Carmichael Numbers}

The FPT can show if a number n is composite, but not if it is prime. But if the FTP shows that n is not composite for a lot of bases a, it is properly prime.

Definition: if n is odd,composite and it holds that $a^{n-1} \equiv 1 \text{ mod } n$, n is called \textbf{pseudoprime} to base a.

Definition: if n is pseudoprime to base a for all a with gcd(a,n) =1, n is called \textbf{Carmichael number}.

\subsection{Miller-Rabin Primitive Test (MRPT)}

For a odd natural number \textit{n}:

\[ s = max\{r \in \mathbb{N}: 2^r \text{ divides } n-1\} \]

So $2^s$ is the highest power of 2 that divides $n-1$. Set $d=\frac{(n-1)}{2^s}$ with d as odd number.

Theorem: If n is prime and gcd(a,n) = 1 it holds either that

\[ a^d \equiv 1 \text{ mod } n \]

or there is one r in \{ 0,1,\dots,s-1 \} with
\[ a^{2^r d} \equiv -1 \text{ mod } n \]

The order of multiplicative groups of integers modulo n is $n-1=2^sd$ if n is prime. The order k of the residue class $a^d + n\mathbb{Z}$ is a power of 2. Hence it holds for $k = 1 = 2^0$:
\[ a^d \equiv 1 \text{ mod } n \]

If $k>1$, then it holds that $k=2^l$ with $1 \leq l \leq s$. The residue class $a^{2^{l-1}d} + n\mathbb{Z}$ is of order 2, but the sole element of order 2 is $-1 + n\mathbb{Z}$, so with $r=l-1$ it holds that:

\[ a^{2^rd} \equiv -1 \mod n \]


\section{Algorithms}
\subsection{Euclidean Algorithm and Extended Euclidean Algorithm}
Find gcd(a,b): Euclidean Algorithm
Recursive algorithm example 
\begin{equation}
    \begin{split}
	a = 1071 \\
	b = 462 \\
	1071 = 2 \times 462 +  147 \\
	462 = 3 \times  147 + 21 \\
	147 = 7 \times 21 + 0 \\ 
	\Rightarrow gcd(a,b) = 21
    \end{split}
        \label{euclidean_example}
\end{equation}

Find inverse: Extended Euclidean Algorithm
\subsection{Handle the big Negatives}
Example:
\[ 2291 \cdot (-47 \cdot 1946 + 4012) = x \text{ mod } 4662\]
Calculate $47 \cdot 1946 = 91462$ and divide $91462/4662 = 19.61\dots$

Then:
\[ -91462 + 20 \cdot 4662 = 1778 \]
\[2291 \cdot (1778 + 4012) = 2291 \cdot 5790 = 2291 \cdot 1128 = 2584248 = 1500 \text{ mod } 4662\]
%\subsection{Elgamal}
%\subsubsection{Parameter}
%
%\begin{table}[h]
	%\centering
	%\begin{tabular}{lc}
	%Primzahl: & p\\
%
	%Primitivwurzel: & a modulo p \\
%
	%Private Key: & x $\in$ \{2,\dots,p-2\} \\
%
	%Public Key: & y = $a^x$ mod p \\
%
	%Nachricht: & m $\in$ \{1,\dots,p-1\} \\
%
	%Verschlüsselung:
%
%
	%\end{tabular}
%\end{table}
%
\section{Hash Function}
\begin{itemize}
    \item it is easy to compute the hash value for any given message
    \item it is infeasible to find a message that has a given hash
    \item it is infeasible to modify a message without hash being changed
    \item it is infeasible to find two different messages with the same hash
\end{itemize}

\section{Block-Cipher-Modes}

\subsection{Electronic Code Book Mode}

\begin{figure}[!ht]
	\begin{center}
		\includegraphics[width=0.5\linewidth]{images/Ecb_encryption.png}
	\end{center}
	\caption{ECB Encryption}
	\label{fig:ecb_en}
\end{figure}
\begin{figure}[!ht]
	\begin{center}
		\includegraphics[width=0.5\linewidth]{images/Ecb_decryption.png}
	\end{center}
	\caption{ECB Decryption}
	\label{fig:ecb_de}
\end{figure}

\subsection{Cipher Block Chaining Mode}

\begin{figure}[!ht]
	\begin{center}
		\includegraphics[width=0.5\linewidth]{images/Cbc_encryption.png}
	\end{center}
	\caption{CBC Encryption}
	\label{fig:cbc_en}
\end{figure}
\begin{figure}[!ht]
	\begin{center}
		\includegraphics[width=0.5\linewidth]{images/Cbc_decryption.png}
	\end{center}
	\caption{CBC decryption}
	\label{fig:cbc_de}
\end{figure}

\subsection{Cipher Feedback Mode}

\begin{figure}[!ht]
	\begin{center}
		\includegraphics[width=0.5\linewidth]{images/Cfb_encryption.png}
	\end{center}
	\caption{CFB Encryption}
	\label{fig:cfb_en}
\end{figure}
\begin{figure}[!ht]
	\begin{center}
		\includegraphics[width=0.5\linewidth]{images/Cfb_decryption.png}
	\end{center}
	\caption{CFB decryption}
	\label{fig:cfb_de}
\end{figure}

\subsection{Output Feedback Mode}

\begin{figure}[!ht]
	\begin{center}
		\includegraphics[width=0.5\linewidth]{images/Ofb_encryption.png}
	\end{center}
	\caption{OFB Encryption}
	\label{fig:ofb_en}
\end{figure}
\begin{figure}[!ht]
	\begin{center}
		\includegraphics[width=0.5\linewidth]{images/Ofb_decryption.png}
	\end{center}
	\caption{OFB decryption}
	\label{fig:ofb_de}
\end{figure}

\subsection{Counter Mode}

\begin{figure}[!ht]
	\begin{center}
		\includegraphics[width=0.5\linewidth]{images/Ctr_encryption.png}
	\end{center}
	\caption{CTR Encryption}
	\label{fig:ctr_en}
\end{figure}
\begin{figure}[!ht]
	\begin{center}
		\includegraphics[width=0.5\linewidth]{images/Ctr_decryption.png}
	\end{center}
	\caption{CTR decryption}
	\label{fig:ctr_de}
\end{figure}

\section{learn by heart}
Given: Alphabet with 26 letters $\mathbb{Z}^{n \times n}_26$

Wanted: Number of invertible matrices


Composite $26 = 2 \cdot 13$


Number of invertible matrices mod 2:


\[2^{n^2} (1 - \frac{1}{2}) (1- \frac{1}{2^2})\dots(1-\frac{1}{2^n})\]


Number of invertible matrices mod 13:


\[13^{n^2} (1 - \frac{1}{13}) (1- \frac{1}{13^2})\dots(1-\frac{2}{13^n})\]


Number of invertible matrices mod 26 is the product of those two numbers:


\[26^{n^2} (1 - \frac{1}{2}) (1- \frac{1}{2^2})\dots(1-\frac{1}{2^n})(1 - \frac{1}{13}) (1- \frac{1}{13^2})\dots(1-\frac{2}{13^n})\]

Group order \#$E(\mathbb{F}_q)$:

\[\text{\#}E(\mathbb{F}_q) = q + 1 -t, t \in \mathbb{Z} \text{ and } |t| \leq q \]

\section{Glossary}
\begin{itemize}
    \item Diffusion: one bit changed in input $\rightarrow$ unpredictable changes in the output
    \item Confusion: if messages and corresponding ciphers are known it is infeasible to find the key
\end{itemize}

%TODO
\section{TODO}
Kettenregel


Münzwurf


\begin{itemize}
    \item \href{http://en.wikipedia.org/wiki/Modular_exponentiation}{Modular exponentiation}: um schnell exponenten zu berechnen (Fermatsche Primzahlzerlegeung) 
\end{itemize}



















\end{document}
