\documentclass[a4paper]{article}
\usepackage[utf8]{inputenc}
\usepackage[T1]{fontenc}
\usepackage{lmodern}
\usepackage{graphicx}
\usepackage{bm}
\usepackage{hyperref}
\usepackage{amsmath}
\usepackage{amssymb}
\usepackage{MnSymbol}
\usepackage{units}
\usepackage{listings}


%macros
%\newcommand{\mod}{\text{ mod }}



\begin{document}
\title{Cryptography}
\section{Primes and Divisibility} 
    %TODO
\section{Modular Arithmetic}
\subsection{Definition}
\begin{equation}
    \begin{split}
	[a \mod N] = [b \mod N] \\
	\text{is defined as: } a \equiv b \mod N
    \end{split}
    \label{modulo_definition}
\end{equation}

\subsection{properties}
Congruence modulo $N$ is an equivalent relation i.e.
\begin{equation}
    \begin{split}
	a \equiv a \mod N  \quad \forall  a \text{ (reflexive)} \\
	a \equiv b \mod N \quad\Rightarrow\quad b \equiv a \mod N \text{ (symmetric)}\\
	a \equiv b \mod N  \quad\wedge\quad  b \equiv c \mod N \quad\Rightarrow\quad a \equiv c \mod N \text{ (transitive)}
    \end{split}
    \label{modulo_equivalent_relation}
\end{equation}
Congruence modulo $N$ respects standard rules of arithmetic i.e.
\begin{equation}
    \begin{split}
	\text{if }a \equiv a' \mod N \quad\wedge\quad b \equiv b' \mod N \\
	\Rightarrow\quad ab \equiv a'b' \mod N \quad\wedge\quad a+b \equiv a'+b' \mod N
    \end{split}
    \label{modulo_arithmetic_rules}
\end{equation}

\textbf{but} does not respect division
\begin{equation}
    ab \equiv cb \mod N \nRightarrow a \equiv c \mod N 
    \label{modulo_arithmetic_division}
\end{equation}

except if $b$ is invertible i.e. it exists $b^{-1}$ that $bb^{-1} \equiv  1 \mod N$ 
\begin{equation}
    \begin{split}
	ab \equiv cb \mod N \\
	\Rightarrow abb^{-1} \equiv cbb^{-1} \mod N \\
	\Rightarrow a \equiv b \mod N
    \end{split}
    \label{modulo_arithmetic_multiplication_invertible}
\end{equation}

\subsubsection{Example}
It is possible to first reduce and then to add or multiply, instead of add or multiply and then reduce.
e.g 
\begin{equation*}
    \begin{split}
	[134092348\cdot134938402 \mod 100] \stackrel{slow}{=} [18094207159547896\mod 100] = 96 \\
	\Leftrightarrow[134092348\cdot134938402 \mod 100] = [134092348 \mod 100]\cdot[134938402 \mod 100] = 48\cdot2 = 96 
    \end{split}
\end{equation*}
But it is not possible to reduce modulo expressions
\begin{equation*}
    \begin{split}
	[15 \cdot 2 = 30] \equiv[3 \cdot 2 = 6] \mod 24 \\
	15 \nequiv 3 \mod 24
    \end{split}
\end{equation*}

\section{Groups}
\subsection{Definition}
A group is a set $\mathbb{G}$ with a binary operation $\circ$ with the following conditions
\begin{equation}
    \begin{split}
	\forall g,h \in \mathbb{G}, g \circ h \in \mathbb{G} \text{ (closure)}\\
	\exists e \in \mathbb{G} \text{ such that } \forall g \in \mathbb{G}, e \circ g = g \circ e = g \text{ (identity)} \\
	\forall g \in \mathbb{G} \exists h \in \mathbb{G} \text{ such that } g \circ h = e = h \circ g \text{ (inverse)}\\
	\forall g_1,g_2,g_3 \exists \mathbb{G} (g_1 \circ g_2) \circ g_3 = g_1 \circ (g_2 \circ g_3) \text{ (associativity)}
    \end{split}
    \label{group_definition}
\end{equation}
If for a Group $\mathbb{G}$ with the operation $\circ$ also the following holds true 
\begin{equation}
    \forall g,h \in \mathbb{G}, g \circ h = h \circ g \text{ (commutativity)}
    \label{group_abelian_defintion}
\end{equation}






















\end{document}
