\documentclass[a4paper]{article}
\usepackage[utf8]{inputenc}
\usepackage[T1]{fontenc}
\usepackage{lmodern}
\usepackage{graphicx}
\usepackage{bm}
\usepackage{hyperref}
\usepackage{amsmath}
\usepackage{amssymb}
\usepackage{MnSymbol}
\usepackage{units}
\usepackage{listings}


%macros
%\newcommand{\mod}{\text{ mod }}



\begin{document}
\title{RU2011}
\section{A1} 

a) q=1 ($k_2=0$ always)

%learn the chain-rules!
alternative 1)

H(M|C)) = H(M,C)-H(C)

=H(M,C,$K_1$)-H($K_1$|M,C) - H(C)

=H(M,C,$K_1$) - H(C)

=H(M,C,$K_1$)-H(M|$K_1$,C)-H(C)

alternative 2)

H(M|C) = = H($K_1 \oplus $C |C) = H($K_1$|C)

b) Derive the distribution of $K_1 \oplus K_2$

\begin{table}[h]
	\begin{tabular}{c|c|c}
$K_1$ & $K_2$ & $K_1 \oplus K_2$ \\
\hline
0 & 0 & 0 \\
0 & 1 & 1 \\
1 & 0 & 1 \\
1 & 1 & 0 
	\end{tabular}
\end{table}

$P(K_1 \oplus K_2 = 0) = pq + (1-p)(1-q) = 1-p-q+2pq$


c) $K = K_1 \oplus K_2$

Theorem 4.13 System perfect secure $\leftrightarrow P(K=K_l=\frac{1}{\mathbb{K}j} \leftrightarrow P(K=0) = \frac{1}{2}$

$1-p-q+2pq = \frac{1}{2}$

$\frac{1}{2} - p = q (1-2p)$

$\frac{(1-2p)\frac{1}{2}}{(1-2p)} = q$

$\rightarrow q = \frac{1}{2}$

d) (see Sec.2.6)

$e(M,K) = C = (M_1 \oplus M_2, M_2 \oplus K)$

$d(C,K)=M=(M_2\oplus C_1, K \oplus C_2)$

$= (K \oplus C_2 \oplus C_1,K \oplus C_2)$

$d(e(M,K)) = M \forall M,K \in \{0,1\}\ast \{0,1\}\ast \{0,1\}$

e) $H(M_1|C_1),H(M_2|C_2)$

$H(M_1|C_1)=-\displaystyle \sum\limits_{(m_1,c_1)\in \{0,1\}^2} P(m_1,c_1) log_2 P(m_1|c_1)$

\begin{table}
	\centering
	\begin{tabular}{c|cc}
		$P(M_1,C_1)$ & $C_1 = 0$ & $=1$ \\

		\hline

		$M_1 = 0$ & $\frac{1}{4}$ & $\frac{1}{4}$ \\

		$M_1 = 1$ & $\frac{1}{4}$ & $\frac{1}{4}$ \\

	\end{tabular}
\end{table}

$H(M_1|C_1)=-4 \frac{1}{4} log_2 \frac{1}{2} =4 \frac{1}{4} log_2 2 = 1$

$H(M_2|C_2)=1$(s.o.)


f) calculate $H(\hat M|\hat C)$

$=H(\hat M,\hat C)-H(\hat C)$

$=H(\hat M,K) - H(\hat C)$

$=H(M_1)+H(M_2)+H(K)-H(C)$

$=1+1+1-2=1$

perfect secrecy $\leftrightarrow H(M|C) = H(M)$

$H(M) = H(M_1) +H(M_2) = 2$

$\rightarrow$ no!

%wichtig für Klausur!



\section{A2} 

p=107 a =2 $x_a=66$ $x_B = 33$

a) shared key is $a^{x_Ax_B}$

$x_Ax_B=66\cdot 33 = 58 \mod (p-1)$

calculate $2^{58} \mod p$

SQM:

$\rightarrow 2^{58} = 75 \mod 107$

b) 103 primitive element

$\leftrightarrow 103^{\frac{p-2}{p_i}}\neq 1 \mod \forall i$

$106 =53 \cdot 2$

$103^2 \equiv 16 \nequiv 1 \mod 107$

$103^{53} = (-4)^{53} = (-2)^{106} = -1 \equiv 106 \nequiv 1 \mod 107$

c)

$103^{58} \equiv (-4)^{58} \equiv 4^{58} \equiv 2^{116} \equiv 2^{10} \equiv 2^7 2^3 \equiv 21\cdot8 \equiv 168 \equiv 61 \mod 107$

\section{A3} 

Public Key (e,n) = (7,$11\cdot13$)

a) Bob's private key : $d=e^{-1} \mod \varphi(n),\varphi(n)=120$

$d\cdot e \equiv 1 \mod \varphi(n)$

$d7+k120=1$

$-17\cdot 7 + 120 = 1 \rightarrow d= -17 \equiv 103 \mod 120$

b) $c_2 = m_1^e \mod n = 110^7 \mod 143 = 33 \mod 143 $
solve by SQM

c)$c_2 = 10$ $ m_2=c_2^d \mod n$

$\rightarrow m_2=10^{103} \mod 143$

$=10 \mod 143$

Solve 103 by SQM

c) public key: (e',n') = (9,253)

private key: d' = 49

Eve knows (e',n'),d'

d) $d'\equiv e'^{-1} \mod \varphi(n')$

$\rightarrow e'\cdot d' \equiv 1 \mod \varphi(n')$

$\rightarrow e'\cdot d' -1 \equiv 0 \mod \varphi(n')$

$\rightarrow 9\cdot49 -1 \equiv 440 \mod \varphi(n')$

$440=220\cdot2=2^3 \cdot 5 \cdot 11$

f) $440 = k \varphi(n) = k(p-1)(q-1), p \neq q;$p,q prime

$=2\cdot(11-1)(23-1)$



\section{A4} 

m=77
p=97
a=5

a) 
a must be a primitve element mod p %help sheet

$p-1 = \displaystyle \prod_{i=1}^k p_i^{t_i} \rightarrow 97-1 = 96 = 2^5 \cdot 3$

$\leftrightarrow a^{\frac{p-1}{p_i}}\mod p  \nequiv 1 \mod p$

$\rightarrow 5^{96/2} = 5^{48} \mod 97 \nequiv 1$

$5^{96/3} = 5^{32} \mod 97 \nequiv 1$

$\rightarrow$ a is a primitive element mod p

b) Elgamal signature without hashed message

$r=a^k \mod p = 5^7 \mod 97 \equiv 78125 \mod 97 \equiv 40$

$K^{-1} \mod (p-1) \equiv 7^{-1} \mod 96 \equiv -41 \equiv 55 \mod 96$

Extended Euclidean Algorithm:

$96 = 7*13+5$

$7=5*1 +2 $

$5 = 2*2 +1 $

$ 1= 5-2*2$

$1=5-2(7-5)=5*3-2*7$

$s=k^{-1}(m-x_A r) \mod (p-1) = 55 (77-8*40) \mod 96 = 75$

check verification:

$v_1 \equiv y_a^r r^s \mod p = v_2 = a^m \mod p$

$\rightarrow y_1 = a^{x_A}\mod p,r=a^k \mod p, s=-rv^{-1}\mod (p-1)$

$v_1 = a^{x_1*r}a^{(u+x_av)(-rv^{-1})}\mod p$

$\equiv a^{x_ar-urv^{-1}-x_ar}\mod p$

$\equiv a^{-urv^{-1}}\mod p$

$v_1 \equiv v_2$

d) It is impossible to find: $m=h^{-1}(-ruv^{-1})$ because a hashfunction is preimage-resistant by definition

e)

$s=x_A^{-1}(m-kr) \mod (p-1)$

verification:

$a^m \equiv y_A^s \cdot r^e$

$\equiv (a^{x_a})^{s} \cdot r^r \equiv a^{x_as}r^r$

$a^{x_ax_a^{-1}(m-kr)}r^r$

$a^{m-kr}a^{kr}\equiv a^m$

f) signature: $s= x_a^{-1}(m+kr) \mod (p-1)$

$a^s \equiv y_a^m r^r \mod p$

$a^{x_am+kr} \equiv a^{x_am}a^{kr}$

g) $s=x_ar+km \mod p-1$
$a^s = \dots = y_a^r r^m \mod p$




\section{A5} 

Lamport protocol/ CR'-auth
a) Security is based on the preimage resistancy of hash-functions

Description: A chooses PW $w$ and calculates $H^t(w) =w_0$

$A\rightarrow B$:$w_0$

In each session i: $w_i = H^{t-i}(w)$

$A\rightarrow B$:$w_i$, B verifies $w_{i-1}=H(w_i)$

b) O can intercept A's message as in man-in-the-middle-attack and he can forward the password to B.
Then B believes that O is A.

c) $A\leftarrow B$: $r_B$

$A\rightarrow B$: $cert_A,r_A, B, S_A(r_A,r_B,B)$

$A \leftarrow B$: $cert_B,A,S_B(r_B,r_A,A)$

d) h(m) = 12 p=137 q=17 a=3 y=136

DSA:$y=a^x\mod p \rightarrow 136 \equiv 3^x \mod 137$

here: $p-1 \equiv a^x \mod p$

$\rightarrow -1 \equiv a^x \mod p$

We know by Euler's criterium that $a^{\frac{p-1}{2}}\mod p = +- 1 \mod p,p>2$

$\rightarrow x=\frac{p-1}{2}=\frac{136}{2} = 68$

k=3

$r= (a^k \mod p ) \mod q$

$\equiv 3^3 \mod 137 \mod 17$

$\equiv 27 \mod 17$

$\equiv 10 \mod 17$

$K^{-1} \mod q \equiv 3^{-1} \equiv 6 $

Extended Euclidean Algorithm:

$17=3\cdot 5 +2$

$3=2\cdot 1 +1$

$1=3-2\cdot 1$

$=3-(17-3\cdot 5)$

$=3\cdot 6 - 17\cdot 1$

$s=K^{-1}(h(m)-xr)\mod q$

$=6(12+68\cdot 10) \mod 17$

$\equiv 4$

\section{A6} 
Elliptic Curves and appl. to cryptography

E: $y^2=x^3+2$,over $\mathbb{F}_5$
\begin{table}[h]
	\centering
	\begin{tabular}{ccc|c}
	$a$ & $a^2$ & $a^3$ & $2+a^3$ \\
	\hline
	0 & 0& 0&2\\
	1&1&1&3\\
	2&4&3&0\\
	3&4&2&4\\
	4&1&4&1
	
	\end{tabular}
\end{table}

$\rightarrow y^2 \in \{0,1,4\}:A$

$\rightarrow 2+x^2 \in \{0,1,2,3,4\}:B$

$C:= A \text{ AND } B = \{0,1,4\}$



$y^2 = 0 \leftrightarrow y=0$

$2+x^2 = 0 \leftrightarrow x=2 \rightarrow (2,0) \in E(\mathbb{F}_5)$

$y^2 = 1 \leftrightarrow y \in \{1,4\}$

$2+x^2=1 \leftrightarrow x=4 \rightarrow (4,1),(4,4)\in E(\mathbb{F}_5)$

$y^2=4 \leftrightarrow y = \{2,3\}$

$2+x^2 = 4 \leftrightarrow x=3 \leftrightarrow (3,2),(3,3) \in E(\mathbb{F}_5)$

$\#E(\mathbb{F}_5)=6$

$E(\mathbb{F}_5)=\{(2,0),(4,1),(4,4),(3,2),(3,3),0\}$

b)

$-(2,0)=(2,0)$

$-(4,1)=(4,4)$

$-(4,4)=(4,1)$

$-(3,2)=(3,3)$

$-(3,3)=(3,2)$

$-\mathcal{O}=\mathcal{O}$


c)
Elgamal Signature Scheme on ell. curves

1) A chooses random integer k: $1\leq k < n,$ gcd(k,n)=1

A computes $R=kP=(r_x,r_y)$

2) $s=-k(m-x_AR)$

3)$A\rightarrow B$: (M,R,S)

Verification:

1) $V_1 = x_A k_i P + SR,V_2=mP$

2) $V_1 = V_2$

sign m=1 with K=2,$P\leq (4,1),y_A=(3,3)$

1) $R=kP=2(4,1)=(r_x,r_y)$

$=(3,3)$

$y_A=x_AP=(3,3)\rightarrow x_A=2$

$2(4,1) \rightarrow x=\left(\frac{3\cdot 4^2+0}{2\cdot1}\right)^2-2\cdot4=(48\cdot3)^2-8\equiv3\mod 5$

$s=-k(M-x_AR)$

$=-2( (4,1) - 2\cdot(3,3))$

$\left[ = - \left[ 2(4,1)+4(4,1) \right] = -6 (4,1) \right]$ DER SUPER TIP!!!

$=-2( (4,1) - 2 (3,2))$

$=-2( (2,0)) = (2,0) + (2,0) = \mathcal{O}$

$\rightarrow (M,R,S) = ( (4,1),(3,3),\mathcal{O})$


d)
Is (2,0) a generator for $\mathbb{F}_5$?

$2\cdot(2,0)=\mathcal{O}$%neutral element

$3\cdot(2,0)=(2,0)$

$\rightarrow (2,0)$ is not a generator of $\mathbb{F}_5$


Groups:

$(G,+) g \in G,|g|,|G|,<g>$

1)$|g| \textbf{|} |G|$

2)$|G|g=0$

3)$\mathcal{O},-g \in <g>$


\end{document}
